\documentclass[11pt, oneside]{article}   	% use "amsart" instead of "article" for AMSLaTeX format
\usepackage{geometry}
\usepackage{enumitem}
\usepackage{algorithm}
\usepackage[noend]{algpseudocode}
             		% See geometry.pdf to learn the layout options. There are lots.
\geometry{letterpaper}                   		% ... or a4paper or a5paper or ...
%\geometry{landscape}                		% Activate for for rotated page geometry
%\usepackage[parfill]{parskip}    		% Activate to begin paragraphs with an empty line rather than an indent
\usepackage{graphicx}				% Use pdf, png, jpg, or eps§ with pdflatex; use eps in DVI mode
								% TeX will automatically convert eps --> pdf in pdflatex
\usepackage{amssymb}

\usepackage{stmaryrd}


\title{CS704: Assignment 3}
\author{}
\date{}							% Activate to display a given date or no date


\begin{document}
\maketitle

The first question is adapted from Nielsen and Nielsen.

\section{Operational Semantics}
In class, we defined the operational semantics of a simple imperative programming language with while loops. In this question, you will extend the language and its semantics.

\begin{enumerate}[label=\bf\Alph*]
  \item Suppose we extend the language with a \texttt{repeat $P$ until $b$} statement, where $b$ is a Boolean expression
  and $P$ is a program.
  Provide the rule(s) defining
   the big-step (natural) semantics of the
   repeat-until statement---i.e., define the $\to$ relation.
  Your semantics should model the intuitive
  meaning of a repeat-until statement: that $P$ is repeatedly
  executed until $b$ is \emph{true}, in which case we exit the
  loop.

  \item Prove that for all programs $P$ and Boolean expressions $b$,
  the programs

  \texttt{repeat $P$ until $b$}

  and

  \texttt{$P$; if $b$ then skip else (repeat $P$ until $b$)}

  are equivalent.

  Recall that two programs $P_1$, $P_2$ are equivalent
  iff  for all states $s,s'$,
  we have
  $$\langle P_1, s \rangle \to s' \emph{ iff } \langle P_2, s \rangle \to s'$$

  \item As above, provide the semantics for
  a for-loop construct of the form

  \texttt{for x = $a$ to $a'$ do $P$}

  where $a$ and $a'$ are arithmetic expressions.

\end{enumerate}
\section{Hoare Logic}

In this question, all variables hold real-world integers---i.e.,
don't worry about machine arithmetic and  overflows.


\begin{enumerate}[label=\bf\Alph*]
\item
Using Hoare logic, give a proof of the following Hoare triple:

$\{x \geq 0 \land y > 0\}$
\begin{verbatim}
r = x;
q = 0;
while (r >= y){
  r = r - y;
  q = q + 1;
}
\end{verbatim}
$\{x = y * q + r \land 0 \leq r < y\}$

You need only provide an annotation of the form $\{P\}$
for every location in the program; you do not need
to show a derivation tree.
Accompany your answer with an English description.

\item Using Hoare logic, give a proof that the following
sequence of statements swaps the values of x and y.

\begin{verbatim}
x = x + y;
y = x - y;
x = x - y;
\end{verbatim}

You may use auxiliary variables to denote the
initial/final values of x and y.
Again, you need only supply annotations of the form $\{P\}$
for every location along the program.
Accompany your answer with an English description.

\item Consider the following Hoare triple.

$\{true\}$
\begin{verbatim}
x = 10;
y = 10;

while (x + y > 0)
  x = x - 1;
  y = y - 1;
  z = x + y;
\end{verbatim}
$\{z=0\}$


Prove that the Hoare triple is valid
by giving an inductive loop invariant.
Show that your answer is indeed an inductive loop invariant.
\end{enumerate}

\section{Type Inference}
In class, we used Robinson's unification algorithm to
solve a set of type equations of the form $\{S_i = T_i\}_{i\in 1\ldots n}.$
In cases where no principal unifier exists,
the algorithm returns \emph{fail}.




For completeness, we supply the unification algorithm below,
in the style of TAPL, as covered in lecture.
The input to the algorithm is a set $C$ of equations
of the form $S=T$.
In the conditionals, we use
 $S = X$ to mean that $S$ is a single variable $X$.
Similarly, $S = S_1 \to S_2$ means that $S$ is of the form
$S_1 \to S_2$, where $S_i$ could be variables, types,
or more complex expressions.

\begin{algorithm}
\caption{Unification Algorithm}\label{alg:unify}
\begin{algorithmic}
\Procedure{unify}{$C$}
\If{$C = \emptyset$} \Return [\ ]
\Else
  \State let $\{S = T\} \cup C' = C$
  \If {$S = T$} \textsc{unify}(C')
  \ElsIf {$S = X$ and $X \not\in FV(T)$}
    \State \Return $\textsc{unify}([X \mapsto T]C') \circ [X \mapsto T]$
  \ElsIf {$T = X$ and $X \not\in FV(S)$}
   \State \Return $\textsc{unify}([X \mapsto S]C') \circ [X \mapsto S]$
  \ElsIf {$S = S_1 \to S_2$ and $T = T_1 \to T_2$}
    \State \Return $\textsc{unify}(C' \cup \{S_1 = T_1, S_2 = T_2\})$
  \Else \State \Return \emph{fail}
  \EndIf
\EndIf
\EndProcedure
\end{algorithmic}
\end{algorithm}
\begin{enumerate}[label=\bf\Alph*]

\item
  Provide an example set of equations where the algorithm fails.
  Ensure that your equations \emph{do not include}
   cases where (1) a variable $X$ appears in both $S_i$ and $T_i$,
  or (2) $S_i$ is a function type and $T_i$ is not (or vice versa).
  In other words, these cases ensure that you do not give a trivial
  answer where the algorithm immediately fails when it considers
  such equation.


\item Prove that \textsc{unify} always terminates.
\end{enumerate}
\end{document}
