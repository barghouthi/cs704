\documentclass{amsart}

\usepackage{hyperref,url,amsmath,amssymb,proof,stmaryrd,tikz-cd,mathabx}

\newtheorem{theorem}{Theorem}[section]
\newtheorem{lemma}[theorem]{Lemma}

\theoremstyle{definition}
\newtheorem{definition}[theorem]{Definition}
\newtheorem{example}[theorem]{Example}
\newtheorem{xca}[theorem]{Exercise}

\theoremstyle{remark}
\newtheorem{remark}[theorem]{Remark}

\numberwithin{equation}{section}

\newcommand{\ifst}{\mathsf{if}}
\newcommand{\thenst}{\mathsf{then}}
\newcommand{\elsest}{\mathsf{else}}

\newcommand{\opsem}[2]{\langle #1 \rangle \to #2}
\newcommand{\trans}{\emph{trans}}
\newcommand{\enc}{\textsf{enc}}
\newcommand{\fv}{\textrm{FV}}
\renewcommand{\paragraph}[1]{\vspace{.08in}\noindent\textbf{#1}~~}


\begin{document}


\title{Logical Encoding of Programs}

\author{Aws Albarghouthi}
\address{University of Wisconsin--Madison}
\email{aws@cs.wisc.edu}

\maketitle

\begin{abstract}
These notes describe the process
of encoding program executions as a formula
in first-order logic.
This allows us to apply various forms of
automated verification by checking satisfiability
of the encodings.
\end{abstract}

\section{Language}
Recall the simple imperative language we used
to present operational semantics in class.
For now, we will use a subset of that language
that does not involve while loops.

\begin{definition}[Language Syntax]
  A program $P$ is defined as follows:
  \begin{align*}
    P ::= &~ x \gets a & \text{assignment}\\
    & \mid \ifst~ b ~ \thenst~ P_1 ~\elsest~ P_2 & \text{conditional}\\
    & \mid P_1; P_2 & \text{sequential composition}
  \end{align*}
  where $a$ is an arithmetic (integer) expression over program variables,
  and $b$ is a Boolean expression over program variables.
  We shall use $V$ to denote the set of all program variables.
  All variables $v \in V$ are assumed to be integer-typed $(\mathbb{Z})$.
\end{definition}

\begin{definition}[Language Semantics]
Recall that a \emph{state} $s: V \to \mathbb{Z}$
of a program $P$ is a map from variables $V$ to integer values.
Given a program $P$, the operational semantics
define a relation $\to$ denoting the execution of $P$ beginning at some state $s$ and ending in state $s'$, denoted as follows:
$$ \opsem{P,s}{s'} $$
\end{definition}

\section{Transition Relations}
We now define \emph{transition relations},
which are relations defining the operational semantics
of a program. We will later show how to define
transition relations in first-order logic.
Given a program $P$, we will define its transition
relation $\trans(V,V')$ over two sets of variables,
the program variables $V$ and a copy of program variables
$V'$. The idea is that the variables $V$ hold the initial
state $s$ of the program, and $V'$ hold the final state $s'$
after the program has completed execution.
This is best seen through an example:

\begin{example}
  Consider the simple program $P$ defined as a single
  statement:
  $$x \gets x + 1$$
  The set of variables $V$ of $P$ is the singleton set $\{x\}$.
  The set $V'$ is $\{x'\}$.
  Therefore the transition relation $\trans(x,x')$
  is over two variables, $x$ and $x'$.
  Concretely, $\trans$ is defined as follows:
  $$\trans = \{(n,n+1) \mid n \in \mathbb{Z}\}$$
  That is, for any number $n$, the pair $(n,n+1)$
  is in the transition relation,
  denoting that if we start executing the program
  with $x = n$, we end up with $x' = n+1$.

  Observe how there is a one-to-one
  correspondence between elements of the transition
  relation and pairs of states $s,s'$ such that
  $\opsem{P,s}{s'}$.
\end{example}

\section{First-Order Theory}
Recall the first-order theories we discussed in class.
To encode the relation $\trans$, we will use
the theorem of \emph{linear integer arithmetic} (LIA).
LIA allows formulas to contain
integer-valued variables, equality, ineqality, addition, and
\emph{no multiplication} (multiplication will only be to replace addition, i.e., $x + x$ will be written as $2x$).
Checking satisfiability of formulas in LIA
is decidable.

\begin{definition}[Linear Integer Arithmetic]
Formally, an LIA formula $\varphi$  is of the following form:
\begin{align*}
  \varphi ::= &~ a_1 = a_2\\
    & \mid  a_1 \leq a_2 \\
    & \mid a_1 < a_2 \\
    & \mid \varphi \land \varphi \\
    & \mid \varphi \lor \varphi \\
    & \mid \neg \varphi\\
    & \mid \exists x \ldotp \varphi\\
    & \mid \forall x \ldotp \varphi
\end{align*}
where $a_1,a_2$ are arithmetic expressions of the form
$c_1x_1 + \ldotp c_nx_n$, where $c_1 \in \mathbb{Z}$
and $x_i$ are first-order variables. For instance, $2x + 3y$.
Note that $\Rightarrow$ and $\iff$ can be written using $\land,\neg$.
\end{definition}

A model $m$ for a formula $\varphi$, denoted $m \models \varphi$
is a mapping from the free variables of $\varphi$,
denoted $\fv(\varphi)$, to integers, that satisfies the
formula.
For example, consider $x + y > 0$.
Let $m = \{x \mapsto 1, y \mapsto 0\}$.
We have  $m \models x + y > 0$

\section{Encoding Transition Relations}
\paragraph{Single assignments}
Let us now consider how we would encode a transition
relation of a single assignment.
Let the assignment be of the form
$$x \gets a$$
We simply transform this into a formula $\varphi$ defined
as follows:
$$\varphi \equiv x' = a \land \bigwedge_{y \neq x, y \in V} y' = y$$
Notice that the variable $x$ on the lhs of the assignment
is encoded as $x'$, denoting the final value of $x$ after
the assignment is performed.
Notice also that the final state of every variable $y$ other than $x$
is the same as its initial value, since it is not modified by the assignment statement.


\begin{example}
  Consider the assignment
  $$x \gets x + y$$
  This is encoded as the transition relation
  $$\trans(x,y,x',y') \equiv x' = x + y \land y' = y$$
  assuming $V = \{x,y\}$
\end{example}

\paragraph{Full Programs}
Now that we have discussed how to construct $\trans$
for a simple program comprised of a single assignment,
we can define the rules for a full (loop-free) programs.

\begin{definition}[Transition-Relation Encoding]
\begin{align*}
  \enc(x \gets a) \triangleq&~  x' = a \land \bigwedge_{y \neq x, y \in V} y' = y\\
  \enc(\ifst~ b ~ \thenst~ P_1 ~\elsest~ P_2) \triangleq&~
    (b \Rightarrow \enc(P_1)) \land (\neg b \Rightarrow \enc(P_2))\\
  \enc(P_1; P_2) \triangleq&~ \exists V'' \ldotp \trans_1(V,V'') \land \trans_2(V'',V')\\
  ~\\
  & \text{where } \trans_1(V,V') \equiv \enc(P_1)\\
        & \hspace{3em} \trans_2(V,V') \equiv \enc(P_2)
\end{align*}
\end{definition}

To illustrate the encoding, let us consider some examples:

\begin{example}
  Consider the following program $P$
  $$\ifst~ x > 0 ~ \thenst~ x \gets x + 1 ~\elsest~ x \gets y$$
  The encoding $\enc(P)$ is a formula over $V$ and $V'$
  defined as follows:
  $$(x > 0 \Rightarrow x' = x + 1 \land y' = y)
  \land (x \leq 0 \Rightarrow x' = y \land y' = y)$$
  The left conjunct defines the transition relation
  of the \emph{then} branch;
  the right conjunct defines the \emph{else} branch.
\end{example}

\begin{example}
  Now consider the following program $P$:
  $$x \gets x + 1; y \gets y + 1$$
  The function $\enc$ encodes this program
  one instruction at a time, and combines the results.
  First, we  encode $x \gets x + 1$ as follows:
  $$\trans_1(x,y,x',y') \equiv
  \enc(x \gets x + 1) \equiv
  x' = x + 1 \land y' = y$$
  Then, we  encode $y \gets y + 1$:
  $$\trans_2(x,y,x',y') \equiv
  \enc(y \gets y + 1) \equiv
  y' = y + 1 \land x' = x$$

  Now that we simply conjoin the two transition relations;
  however, we have to be careful about how the variable
  names: the output variables of $\trans_1$ need be the same as
  the input variables of $\trans_2$.
  Therefore we conjoin them as follows:
  $$\trans_1(x,y,x'',y'') \land \trans_2(x'',y'',x',y')$$
  Notice that we renamed the outputs of $\trans_1$
  and inputs of $\trans_2$ so that they actually match;
  these variables are $V'' = \{x'',y''\}$.
  Intuitively, $V''$ defines the state of the program
  after executing the first instruction;
  but since we only care about the final state,
  we \emph{quantify the variables $V''$ away}.
  Finally, we arrive at the following encoding

  $$\exists x'',y'' \ldotp
  (x'' = x + 1 \land y'' = y) \land
  (y' = y'' + 1 \land x' = x'')$$
  The first conjunct is $\trans_1(x,y,x'',y'')$,
  and the second is
  $\trans_2(x'',y'',x',y')$.
  Notice that the free variables of this formula
  are $x,y$ and $x',y'$, defining the initial states
  and final states, respectively.
\end{example}


\section{Soundness and Completeness}
We need to show that our encoding
is sound and complete.
Soundness ensures that
the transition relation does not over-approximate
the operational semantics relation $\to$.
Completeness, on the other hand,
ensures that every element of $\to$ is also in
the encoding.

\begin{theorem}[Soundness]
  Fix a program $P$ with variables $V$.
  Let $m \models \enc(P)$.
  Let $$s = \{v \mapsto m(v) \mid v \in V\}$$
  $$s' = \{v' \mapsto m(v') \mid v' \in V'\}$$
  Then, $\opsem{P,s}{s'}$.
\end{theorem}

\begin{theorem}[Completeness]
  Fix a program $P$ with variables $V$.
  Let $s,s'$ be such that
  $\opsem{P,s}{s'}$.
  Let
  $$m = \{v\mapsto s(v) \mid v \in V\} \cup \{v'\mapsto s'(v') \mid v' \in V'\}$$
  Then, $m \models \enc(P)$.
\end{theorem}

\begin{proof}
  Both proofs proceed by structural induction
  on the program.
\end{proof}

\section{Verification}
Now that we have defined the encoding,
we can use it to check if a Hoare triple
holds.
Suppose we are given a Hoare triple
$\{\phi\} P \{\psi\}$,
where $\phi$ and $\psi$ are LIA formulas
over program variables.
We would like to check if the Hoare triple is valid;
informally, for any state $s$ satisfying $\phi$,
for any state $s'$ such that $\opsem{P,s}{s'}$,
we want to make sure that $s'$ satisfies $\psi$.

To automatically check if the Hoare triple
is valid, we encode \emph{all} executions starting at
$\phi$ and check whether they end up in $\psi$.
Formally, we check validity of the following formula:
$$ (\phi \land \enc(P)) \Rightarrow \psi' \text{ is VALID}$$
$$\emph{iff}$$
$$\{\phi\} P \{\psi\} \text{ is VALID}$$
Intuitively, $\phi \land \enc(P)$ defines
the set executions of $P$ constrained to the ones
starting at $\phi$. The implication ensures
that all final states are contained in $\psi'$.
Notice that we use a primed version of $\psi$,
since final states in the encoding are over primed variables.

\begin{example}
  Let us consider a simple example.
  To check validity of
  $$\{x > 0\} x \gets x + 1 \{x > 1\}$$
  we check validity of the following formula:
  $$(x > 0 \land x' = x + 1) \Rightarrow x' > 1$$
  which is valid.
\end{example}

\begin{example}
  Here is another example, but this time the Hoare triple
  is invalid:
  $$\{x > 0\} x \gets x + y \{x > 1\}$$
  The formula $\Psi$
    $$\Psi \equiv (x > 0 \land x' = x + y \land y'=y) \Rightarrow x' > 1$$
    is not valid.
    Here is a counterexample; i.e., a model $m$ for the negation
    of the formula, $\neg\Psi$:
    $$m = \{x \mapsto 0, y \mapsto 0, x' \mapsto 0, y' \mapsto 0\}$$
    In other words, $m$ says that if $x = y = 0$
    in the initial state, the final state will be such that
    $x = 0$, therefore invalidating the postcondition,
    which specifies the $x > 1$ when the program terminates.
\end{example}
\end{document}
